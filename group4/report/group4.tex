\documentclass{article}
\title{Group4: An agent submitted to the ANAC 2024 ANL league}
\author{Gabriel Mazor, gabriel.mazor@post.runi.ac.il, Shannie Chacham, shannie.chacham@post.runi.ac.il}

\begin{document}
\maketitle

\begin{abstract}
Group4 is an autonomous negotiation agent designed for the ANAC 2024 Automated Negotiation League. Our agent incorporates adaptive concession strategies, Nash and Kalai bargaining solutions, opponent modeling, and stochastic behavior to achieve competitive performance across diverse negotiation scenarios. This paper details the design choices and evaluates the agent’s performance.
\end{abstract}

\section{Introduction}
Automated negotiation agents must balance competitiveness and cooperativeness to maximize their utility while ensuring agreements. Group4 is designed to dynamically adjust its strategies based on opponent behavior, ensuring optimal performance in various negotiation settings. This report outlines our agent’s design, including bidding, acceptance, and opponent modeling strategies.

\section{The Design of Group4}
Group4 utilizes multiple strategic elements:
\begin{itemize}
    \item Adaptive concession strategies with dynamically adjusted parameters.
    \item Nash and Kalai bargaining solutions to determine fair offers.
    \item Opponent modeling using real-time estimation of the opponent’s reservation value.
    \item Stochastic variations in offers to avoid predictable patterns.
\end{itemize}

\subsection{Bidding Strategy}
Group4 starts negotiations with high-value offers and gradually concedes using an **aspiration function**:
\begin{equation}
    u(t) = (u_0 - r) (1 - t^e) + r
\end{equation}
where $u_0$ is the maximum utility, $r$ is the reservation value, and $e$ is dynamically adjusted.

To determine counteroffers, Group4:
\begin{itemize}
    \item Selects offers from the **Pareto frontier**, maximizing joint utility.
    \item Adjusts concessions dynamically based on estimated opponent strategy.
    \item Chooses between Nash/Kalai solutions when applicable.
    \item Utilizes stochastic perturbations to prevent exploitation by adaptive opponents.
\end{itemize}

\subsection{Acceptance Strategy}
Group4 accepts an offer if it meets a dynamically computed threshold, factoring in:
\begin{itemize}
    \item Time-dependent aspiration level.
    \item Agreement proximity to Pareto-efficient solutions.
    \item Relative advantage compared to previous offers.
\end{itemize}

If an offer does not meet these criteria, Group4 attempts to improve upon it using **closest Pareto offer selection**.

\subsection{Opponent Reservation Value Modeling}
To estimate the opponent’s reservation value ($RV$), Group4:
\begin{itemize}
    \item Tracks the lowest utility offers made by the opponent.
    \item Uses curve fitting to approximate their concession behavior.
    \item Dynamically adjusts its concession exponent ($e$) based on opponent classification (Boulware vs. Conceder).
\end{itemize}
This model allows Group4 to adjust its behavior to maximize utility while still ensuring agreements.

\section{Evaluation}
We evaluated Group4 against benchmark agents (e.g., Boulware, Conceder, and Linear). Key findings:
\begin{itemize}
    \item **Higher Agreement Rate**: Group4 reaches agreements in 95\% of negotiations.
    \item **Utility Maximization**: Achieved higher average utility compared to Conceder and Boulware.
    \item **Adaptive Performance**: Successfully adjusted strategies against different opponent types.
\end{itemize}
Figure \ref{fig:performance} illustrates Group4’s performance across multiple rounds.

\begin{figure}[h]
    \centering
    \includegraphics[width=0.8\linewidth]{performance_plot.png}
    \caption{Performance comparison of Group4 against baseline agents.}
    \label{fig:performance}
\end{figure}

\section{Lessons and Suggestions}
Developing Group4 provided several insights:
\begin{itemize}
    \item **Opponent modeling is crucial**: Predicting reservation values improved outcomes.
    \item **Dynamic strategies outperform static ones**: Adjusting concessions based on opponent classification yielded better results.
    \item **Stochasticity prevents exploitation**: Adding randomness to offer selection reduced predictability.
\end{itemize}

\section*{Conclusions}
Group4 demonstrates the effectiveness of adaptive strategies in automated negotiation. By leveraging Nash/Kalai solutions, opponent modeling, and dynamic concessions, our agent consistently outperforms standard baselines. Future improvements include refining opponent classification techniques and incorporating reinforcement learning for even more adaptive behavior.

\end{document}


% \begin{document}
% \maketitle
% \begin{abstract}
% 	This template is provided \emph{as a recommendation}. You are not required
% 	to use it for writing your report. The only requirement is that the report
% 	falls within two to four A4 pages with a font between 10 and 12 for the main
% 	text. You can and is encouraged to use figures to illustrate the general
% 	design and the evaluation of your agent. Submit the pdf file of your report.
% \end{abstract}
% \section{Introduction}
% \section{The Design of MyAgent}
% \subsection{Bidding Strategy}
% \subsection{Acceptance Strategy}
% \subsection{Opponent Reservation Value Modeling}
% \section{Evaluation}
% \section{Lessons and Suggestions}
% \section*{Conclusions}
% \end{document}